\documentclass[journal, 11pt, twocolumn]{IEEEtran}
\usepackage[utf8]{inputenc}
\usepackage{hyperref}
\usepackage{graphicx}
\usepackage{amssymb}
\usepackage{amsmath}
\graphicspath{{figs/}}

\font\twelve=cmr12 at 15pt 
\title{\twelve AI1110 Assignment 1}
\author{Abhinav Yadav, cs21btech11002}

\begin{document}
    \maketitle
    \textbf{ICSE class 10 paper 2019 Q11 (b)}\\
    The product of two consecutive natural numbers which are multiples of 3 is 
    equal to 810. Find the two numbers.\\\\
    \textbf{Solution:}\\
    Let the two consecutive natural numbers which are multiples of $3$ be $3n$ and $3n+3$
    \hspace{5pt} $\exists \hspace{2pt} n \in \mathbb{N}$\\
    \textbf{According to the question:}
    \begin{align}
        &&3n(3n+3) &= 810\\
        &\implies & 9n(n+1) &= 810\\
        &\implies & n(n+1) &= 90\\
        \label{eq4}
        &\implies & n^2+n-90 &= 0\hspace{25pt}\\
        &\implies & (n+10)(n-9) &= 0\\
        &\implies & n=-10 \hspace{15pt} &or \hspace{15pt} n=9
        \intertext{discarding $n=-10$ as $n \in \mathbb{N}$}
        &\implies & n &= 9\\
        &\implies & 3n &= 27\\
        &\implies & 3n+3 &= 30
    \end{align}
    The two numbers are:
    \fbox{$27, 30$}\\

    Plot of \autoref{eq4} is:
    \begin{figure}[h]
        \centering
        \includegraphics[width=\columnwidth]{plot.png}
        \caption{Plot showing the polynomial in \autoref{eq4}}
        \label{Fig1}
    \end{figure}\\
    It can be easily verified by observing the plot in \autoref{Fig1} that the roots of \autoref{eq4} are 9 and -10.\\

    The output of the program used to find and verify these numbers is:
    \begin{figure}[h]
        \includegraphics[width=\columnwidth]{output.png}
        \caption{Output of the python program}
        \label{Fig2}
    \end{figure}
\end{document}