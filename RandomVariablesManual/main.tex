\documentclass[journal,12pt,twocolumn]{IEEEtran}
\usepackage{setspace}
\usepackage{gensymb}
\usepackage{caption}
%\usepackage{multirow}
%\usepackage{multicolumn}
%\usepackage{subcaption}
%\doublespacing
\singlespacing
\usepackage{csvsimple}
\usepackage{amsmath}
\usepackage{multicol}
%\usepackage{enumerate}
\usepackage{amssymb}
%\usepackage{graphicx}
\usepackage{newfloat}
%\usepackage{syntax}
\usepackage{listings}
% \usepackage{iithtlc}
\usepackage{color}
\usepackage{tikz}
\usetikzlibrary{shapes,arrows}



%\usepackage{graphicx}
%\usepackage{amssymb}
%\usepackage{relsize}
%\usepackage[cmex10]{amsmath}
%\usepackage{mathtools}
%\usepackage{amsthm}
%\interdisplaylinepenalty=2500
%\savesymbol{iint}
%\usepackage{txfonts}
%\restoresymbol{TXF}{iint}
%\usepackage{wasysym}
\usepackage{amsthm}
\usepackage{mathrsfs}
\usepackage{txfonts}
\usepackage{stfloats}
\usepackage{cite}
\usepackage{cases}
\usepackage{mathtools}
\usepackage{caption}
\usepackage{enumerate}	
\usepackage{enumitem}
\usepackage{amsmath}
%\usepackage{xtab}
\usepackage{longtable}
\usepackage{multirow}
%\usepackage{algorithm}
%\usepackage{algpseudocode}
\usepackage{enumitem}
\usepackage{mathtools}
\usepackage{hyperref}
%\usepackage[framemethod=tikz]{mdframed}
\usepackage{listings}
    %\usepackage[latin1]{inputenc}                                 %%
    \usepackage{color}                                            %%
    \usepackage{array}                                            %%
    \usepackage{longtable}                                        %%
    \usepackage{calc}                                             %%
    \usepackage{multirow}                                         %%
    \usepackage{hhline}                                           %%
    \usepackage{ifthen}                                           %%
  %optionally (for landscape tables embedded in another document): %%
    \usepackage{lscape}     


\usepackage{url}
\def\UrlBreaks{\do\/\do-}


%\usepackage{stmaryrd}


%\usepackage{wasysym}
%\newcounter{MYtempeqncnt}
\DeclareMathOperator*{\Res}{Res}
%\renewcommand{\baselinestretch}{2}
\renewcommand\thesection{\arabic{section}}
\renewcommand\thesubsection{\thesection.\arabic{subsection}}
\renewcommand\thesubsubsection{\thesubsection.\arabic{subsubsection}}

\renewcommand\thesectiondis{\arabic{section}}
\renewcommand\thesubsectiondis{\thesectiondis.\arabic{subsection}}
\renewcommand\thesubsubsectiondis{\thesubsectiondis.\arabic{subsubsection}}

% correct bad hyphenation here
\hyphenation{op-tical net-works semi-conduc-tor}

%\lstset{
%language=C,
%frame=single, 
%breaklines=true
%}

%\lstset{
	%%basicstyle=\small\ttfamily\bfseries,
	%%numberstyle=\small\ttfamily,
	%language=Octave,
	%backgroundcolor=\color{white},
	%%frame=single,
	%%keywordstyle=\bfseries,
	%%breaklines=true,
	%%showstringspaces=false,
	%%xleftmargin=-10mm,
	%%aboveskip=-1mm,
	%%belowskip=0mm
%}

%\surroundwithmdframed[width=\columnwidth]{lstlisting}
\def\inputGnumericTable{}                                 %%
\lstset{
%language=C,
frame=single, 
breaklines=true,
columns=fullflexible
}
 

\begin{document}
%
\tikzstyle{block} = [rectangle, draw,
text width=3em, text centered, minimum height=3em]
\tikzstyle{sum} = [draw, circle, node distance=3cm]
\tikzstyle{input} = [coordinate]
\tikzstyle{output} = [coordinate]
\tikzstyle{pinstyle} = [pin edge={to-,thin,black}]

\theoremstyle{definition}
\newtheorem{theorem}{Theorem}[section]
\newtheorem{problem}{Problem}
\newtheorem{proposition}{Proposition}[section]
\newtheorem{lemma}{Lemma}[section]
\newtheorem{corollary}[theorem]{Corollary}
\newtheorem{example}{Example}[section]
\newtheorem{definition}{Definition}[section]
%\newtheorem{algorithm}{Algorithm}[section]
%\newtheorem{cor}{Corollary}
\newcommand{\BEQA}{\begin{eqnarray}}
        \newcommand{\EEQA}{\end{eqnarray}}
\newcommand{\define}{\stackrel{\triangle}{=}}

\bibliographystyle{IEEEtran}
%\bibliographystyle{ieeetr}

\providecommand{\nCr}[2]{\,^{#1}C_{#2}} % nCr
\providecommand{\nPr}[2]{\,^{#1}P_{#2}} % nPr
\providecommand{\mbf}{\mathbf}
\providecommand{\pr}[1]{\ensuremath{\Pr\left(#1\right)}}
\providecommand{\qfunc}[1]{\ensuremath{Q\left(#1\right)}}
\providecommand{\sbrak}[1]{\ensuremath{{}\left[#1\right]}}
\providecommand{\lsbrak}[1]{\ensuremath{{}\left[#1\right.}}
\providecommand{\rsbrak}[1]{\ensuremath{{}\left.#1\right]}}
\providecommand{\brak}[1]{\ensuremath{\left(#1\right)}}
\providecommand{\lbrak}[1]{\ensuremath{\left(#1\right.}}
\providecommand{\rbrak}[1]{\ensuremath{\left.#1\right)}}
\providecommand{\cbrak}[1]{\ensuremath{\left\{#1\right\}}}
\providecommand{\lcbrak}[1]{\ensuremath{\left\{#1\right.}}
\providecommand{\rcbrak}[1]{\ensuremath{\left.#1\right\}}}
\providecommand{\gauss}[2]{\mathcal{N}\ensuremath{\left(#1,#2\right)}}
\theoremstyle{remark}
\newtheorem{rem}{Remark}
\newcommand{\sgn}{\mathop{\mathrm{sgn}}}
\providecommand{\abs}[1]{\left\vert#1\right\vert}
\providecommand{\res}[1]{\Res\displaylimits_{#1}}
\providecommand{\norm}[1]{\left\Vert#1\right\Vert}
\providecommand{\mtx}[1]{\mathbf{#1}}
\providecommand{\mean}[1]{E\left[ #1 \right]}
\providecommand{\fourier}{\overset{\mathcal{F}}{ \rightleftharpoons}}
%\providecommand{\hilbert}{\overset{\mathcal{H}}{ \rightleftharpoons}}
\providecommand{\system}{\overset{\mathcal{H}}{ \longleftrightarrow}}
%\newcommand{\solution}[2]{\textbf{Solution:}{#1}}
\newcommand{\solution}{\noindent \textbf{Solution: }}
\newcommand{\myvec}[1]{\ensuremath{\begin{pmatrix}#1\end{pmatrix}}}
\providecommand{\dec}[2]{\ensuremath{\overset{#1}{\underset{#2}{\gtrless}}}}
\DeclarePairedDelimiter{\ceil}{\lceil}{\rceil}
%\numberwithin{equation}{section}
%\numberwithin{problem}{subsection}
%\numberwithin{definition}{subsection}
\makeatletter
\@addtoreset{figure}{section}
\makeatother

\let\StandardTheFigure\thefigure
%\renewcommand{\thefigure}{\theproblem.\arabic{figure}}
\renewcommand{\thefigure}{\thesection}


%\numberwithin{figure}{subsection}

%\numberwithin{equation}{subsection}
%\numberwithin{equation}{section}
%\numberwithin{equation}{problem}
%\numberwithin{problem}{subsection}
\numberwithin{problem}{section}
%%\numberwithin{definition}{subsection}
%\makeatletter
%\@addtoreset{figure}{problem}
%\makeatother
\makeatletter
\@addtoreset{table}{section}
\makeatother

\let\StandardTheFigure\thefigure
\let\StandardTheTable\thetable
\let\vec\mathbf
\numberwithin{equation}{section}

\vspace{3cm}

\author{Abhinav Yadav \\ cs21btech11002}
\title{AI1110 \\ Random Variables}

% make the title area
\maketitle

\tableofcontents

\bigskip

\renewcommand{\thefigure}{\theenumi}
\renewcommand{\thetable}{\theenumi}

\begin{abstract}
    This manual provides a simple introduction to the generation of random numbers
\end{abstract}
%%
\section{Uniform Random Numbers}
Let $U$ be a uniform random variable between 0 and 1.
\begin{enumerate}[label=\thesection.\arabic*
        ,ref=\thesection.\theenumi]
    \item Generate $10^6$ samples of $U$ using a C program and save into a file called uni.dat .\\
          \solution Download the following files.
          \begin{lstlisting}
wget https://raw.githubusercontent.com/abhinavydv/AI1110-Assignments/master/RandomVariablesManual/codes/coeffs.h
wget https://raw.githubusercontent.com/abhinavydv/AI1110-Assignments/master/RandomVariablesManual/codes/uni_dat.c
        \end{lstlisting}

          Compile and execute the program as follows.
          \begin{lstlisting}
$ gcc uni_dat.c -lm -o uni_dat
$ ./uni_dat
        \end{lstlisting}

          %
    \item
          Load the uni.dat file into python and plot the empirical CDF of $U$ using the samples in uni.dat. The CDF is defined as
          \begin{align}
              F_{U}(x) = \pr{U \le x}
          \end{align}
          \\
          \solution  Download the following code
          \begin{lstlisting}
wget https://raw.githubusercontent.com/abhinavydv/AI1110-Assignments/master/RandomVariablesManual/codes/cdf_plot_uni.py
\end{lstlisting}
          and run it as follows to get \autoref{fig:uni_cdf}
          \begin{lstlisting}
    $ python cdf_plot_uni.py
\end{lstlisting}
          \begin{figure}
              \centering
              \includegraphics[width=\columnwidth]{./figs/uni_cdf}
              \caption{The CDF of $U$}
              \label{fig:uni_cdf}
          \end{figure}

          %
    \item
          Find a  theoretical expression for $F_{U}(x)$.\\
          \solution
          The pdf of U is given by
          \begin{align}
              P_U(x) & = \begin{cases}
                  1, & 0 \le x \le 1    \\
                  0, & \text{otherwise} \\
              \end{cases}    \\
              \intertext{Now,}
              F_U(x) & = \int_{-\infty}^{x} P_U(t) dt \\
              F_U(x) & = \begin{cases}
                  \int_{-\infty}^{x} 0, & x < 0            \\
                  \int_{0}^{x}1,        & 0 \le x \le 1    \\
                  \int_0^11dx,          & \text{otherwise}
              \end{cases}    \\
              F_U(x) & = \begin{cases}
                  0, & x < 0            \\
                  x, & 0 \le x \le 1    \\
                  1, & \text{otherwise}
              \end{cases}
          \end{align}


    \item
          The mean of $U$ is defined as
          %
          \begin{equation}
              E\sbrak{U} = \frac{1}{N}\sum_{i=1}^{N}U_i
          \end{equation}
          %
          and its variance as
          %
          \begin{equation}
              \text{var}\sbrak{U} = E\sbrak{U- E\sbrak{U}}^2
          \end{equation}

          Write a C program to  find the mean and variance of $U$.\\
          \solution Download the following files.
          \begin{lstlisting}
wget https://raw.githubusercontent.com/abhinavydv/AI1110-Assignments/master/RandomVariablesManual/codes/coeffs.h
wget https://raw.githubusercontent.com/abhinavydv/AI1110-Assignments/master/RandomVariablesManual/codes/mean_var_uni.c
        \end{lstlisting}

          Compile and execute the program as follows.
          \begin{lstlisting}
$ gcc mean_var_uni.c -lm -o mean_var_uni
$ ./mean_var_uni
        \end{lstlisting}

          After running the program, we get,
          \begin{align}
              \mu = 0.500007 \\
              \sigma = 0.083301
          \end{align}

    \item Verify your result theoretically given that
\end{enumerate}
%
\begin{equation}
    E\sbrak{U^k} = \int_{-\infty}^{\infty}x^kdF_{U}(x)
\end{equation}
\solution Integrating, we get
\begin{align}
    E[U^k]        & = \int_{-\infty}^{\infty}x^kP_U(x)dx      \\
                  & = \int_{0}^1x^kdx                         \\
                  & = \frac{1}{k+1}
    \intertext{Now,}
    E[U]          & = \frac{1}{2}                             \\
    \intertext{This agrees with the experimental value of 0.500007}
    E[U^2]        & = \frac{1}{3}                             \\
    E[(U-E(U))^2] & = E[(U-1/2)^2]                            \\
                  & = E[U^2] + E(1/4) - 2E(\frac{U}{2})       \\
                  & = \frac{1}{3} + \frac{1}{4} - \frac{1}{2} \\
                  & = \frac{1}{12}
\end{align}
This agrees with the experimental value 0.083301
\section{Central Limit Theorem}
%
\begin{enumerate}[label=\thesection.\arabic*
        ,ref=\thesection.\theenumi]

    %
    \item
          Generate $10^6$ samples of the random variable
          %
          \begin{equation}
              X = \sum_{i=1}^{12}U_i -6
          \end{equation}
          %
          using a C program, where $U_i, i = 1,2,\dots, 12$ are  a set of independent uniform random variables between 0 and 1
          and save in a file called gau.dat\\
          \solution Download the following files.
          \begin{lstlisting}
wget https://raw.githubusercontent.com/abhinavydv/AI1110-Assignments/master/RandomVariablesManual/codes/coeffs.h
wget https://raw.githubusercontent.com/abhinavydv/AI1110-Assignments/master/RandomVariablesManual/codes/gau_dat.c
        \end{lstlisting}

          Compile and execute the program as follows.
          \begin{lstlisting}
$ gcc gau_dat.c -lm -o gau_dat
$ ./gau_dat
        \end{lstlisting}


          %
    \item
          Load gau.dat in python and plot the empirical CDF of $X$ using the samples in gau.dat. What properties does a CDF have?
          \\
          \solution  Download the following code
          \begin{lstlisting}
wget https://raw.githubusercontent.com/abhinavydv/AI1110-Assignments/master/RandomVariablesManual/codes/cdf_plot_gau.py
\end{lstlisting}
          and run it as follows to get \autoref{fig:gauss_cdf}
          \begin{lstlisting}
    $ python cdf_plot_gau
\end{lstlisting}
          \begin{figure}
              \centering
              \includegraphics[width=\columnwidth]{./figs/gauss_cdf}
              \caption{The CDF of $X$}
              \label{fig:gauss_cdf}
          \end{figure}
          The properties of a CDF are:
          \begin{itemize}
              \item $0 \le F_X(x) \le 1$
              \item $ \forall x \in (-\infty, \infty),  \frac{d}{dx}F_{X}(x) \ge 0$
              \item $\forall a \in (-\infty, \infty)$, \[\lim_{x \to a^+} F_{X}(x) = F_X(a)\]
          \end{itemize}

    \item
          Load gau.dat in python and plot the empirical PDF of $X$ using the samples in gau.dat. The PDF of $X$ is defined as
          \begin{align}
              p_{X}(x) = \frac{d}{dx}F_{X}(x)
          \end{align}
          What properties does the PDF have?
          \\
          \solution  Download the following code
          \begin{lstlisting}
wget https://raw.githubusercontent.com/abhinavydv/AI1110-Assignments/master/RandomVariablesManual/codes/pdf_plot_gau.py
\end{lstlisting}
          and run it as follows to get \autoref{fig:gauss_pdf}
          \begin{lstlisting}
    $ python pdf_plot_gau.py
\end{lstlisting}

          \begin{figure}
              \centering
              \includegraphics[width=\columnwidth]{./figs/gauss_pdf}
              \caption{The PDF of $X$}
              \label{fig:gauss_pdf}
          \end{figure}

          The PDF has following properties:
          \begin{itemize}
              \item $P_X(x) \ge 0, \forall x \in \mathbb{R}$
              \item $\int_{-\infty}^\infty P_X(x) = 1$
          \end{itemize}

    \item Find the mean and variance of $X$ by writing a C program.\\
          \solution Download the following files.
          \begin{lstlisting}
wget https://raw.githubusercontent.com/abhinavydv/AI1110-Assignments/master/RandomVariablesManual/codes/coeffs.h
wget https://raw.githubusercontent.com/abhinavydv/AI1110-Assignments/master/RandomVariablesManual/codes/mean_var_gau.c
  \end{lstlisting}

          Compile and execute the program as follows.
          \begin{lstlisting}
$ gcc mean_var_gau.c -lm -o mean_var_gau
$ ./mean_var_gau
  \end{lstlisting}

          After running the program, we get,
          \begin{align}
              \mu = 0.000294 \\
              \sigma = 0.999560
          \end{align}

    \item Given that
          \begin{align}
              \label{eq:exp_dist_given}
              p_{X}(x) = \frac{1}{\sqrt{2\pi}}\exp\brak{-\frac{x^2}{2}}, -\infty < x < \infty,
          \end{align}
          repeat the above exercise theoretically.\\
          \solution The mean is given by
          \begin{align}
              \mu & = E(X)                                                                       \\
                  & =\int_{-\infty}^{\infty} x \frac{1}{\sqrt{2\pi}}\exp\brak{-\frac{x^2}{2}} dx \\
                  & = \frac{1}{\sqrt{2\pi}} \times 0 = 0
          \end{align}

          Variance is given by
          \begin{align}
              \sigma^2 & = E(X^2)                                                                                                                                     \\
                       & = \int_{-\infty}^\infty x\times x \frac{1}{\sqrt{2\pi}}\exp\brak{-\frac{x^2}{2}} dx                                                          \\
              \intertext{Integrating by parts, we get}
                       & = \frac{1}{\sqrt{2\pi}} \brak{\left[-x\exp\brak{-\frac{x^2}{2}}\right]_{-\infty}^{\infty}+\int_{-\infty}^\infty \exp\brak{-\frac{x^2}{2}}dx} \\
                       & = \frac{1}{\sqrt{2\pi}} \brak{0 + \sqrt{2\pi}}                                                                                               \\
                       & = 1
          \end{align}

\end{enumerate}
\section{From Uniform to Other}
\begin{enumerate}[label=\thesection.\arabic*
        ,ref=\thesection.\theenumi]
    %
    \item
          Generate samples of
          %
          \begin{equation}
              V = -2\ln\brak{1-U}
          \end{equation}
          %
          and plot its CDF.\\
          \solution To generate samples of V download the following files.
          \begin{lstlisting}
wget https://raw.githubusercontent.com/abhinavydv/AI1110-Assignments/master/RandomVariablesManual/codes/coeffs.h
wget https://raw.githubusercontent.com/abhinavydv/AI1110-Assignments/master/RandomVariablesManual/codes/exp_dat.c
        \end{lstlisting}

          and then compile and execute the program as follows.
          \begin{lstlisting}
$ gcc exp_dat.c -lm -o exp_dat
$ ./exp_dat
        \end{lstlisting}

          To plot the CDF download the following code
          \begin{lstlisting}
wget https://raw.githubusercontent.com/abhinavydv/AI1110-Assignments/master/RandomVariablesManual/codes/cdf_plot_exp.py
\end{lstlisting}
          and run it as follows to get \autoref{fig:exp_cdf}
          \begin{lstlisting}
  $ python cdf_plot_exp.py
\end{lstlisting}

          \begin{figure}
              \centering
              \includegraphics[width=\columnwidth]{./figs/exp_cdf}
              \caption{The PDF of $V$}
              \label{fig:exp_cdf}
          \end{figure}

    \item Find a theoretical expression for $F_V(x)$.\\
          \solution
          \begin{align}
              F_V(x) & = \pr{V \le x}                         \\
                     & = \pr{-2\ln(1-U) \le x}                \\
                     & = \pr{\ln(1-U) \le -\frac{x}{2}}       \\
                     & = \pr{U \le 1-\exp\brak{-\frac{x}{2}}} \\
                     & = F_U(1-\exp\brak{-\frac{x}{2}})       \\
                     & = \begin{cases}
                  0,                         & x<0     \\
                  1-\exp\brak{-\frac{x}{2}}, & x \ge 0
              \end{cases}
          \end{align}

          %
          %\item
          %Generate the Rayleigh distribution from Uniform. Verify your result through graphical plots.
\end{enumerate}

\section{Triangular Distribution}
\begin{enumerate}[label=\thesection.\arabic*
        ,ref=\thesection.\theenumi]
    %
    \item Generate
          \begin{align}
              T = U_1+U_2
          \end{align}
          \solution
          Download the following files.
          \begin{lstlisting}
wget https://raw.githubusercontent.com/abhinavydv/AI1110-Assignments/master/RandomVariablesManual/codes/coeffs.h
wget https://raw.githubusercontent.com/abhinavydv/AI1110-Assignments/master/RandomVariablesManual/codes/tri_dat.c
\end{lstlisting}

          Compile and execute the program as follows.
          \begin{lstlisting}
$ gcc tri_dat.c -lm -o tri_dat
$ ./tri_dat
\end{lstlisting}


    \item Find the CDF of $T$.\\
          \solution  Download the following code
          \begin{lstlisting}
wget https://raw.githubusercontent.com/abhinavydv/AI1110-Assignments/master/RandomVariablesManual/codes/cdf_plot_tri.py
\end{lstlisting}
          and run it as follows to get \autoref{fig:tri_cdf}
          \begin{lstlisting}
    $ python cdf_plot_tri.py
\end{lstlisting}
          \begin{figure}
              \centering
              \includegraphics[width=\columnwidth]{./figs/tri_cdf}
              \caption{The CDF of $T$}
              \label{fig:tri_cdf}
          \end{figure}

    \item Find the PDF of $T$.\\
          \solution  Download the following code
          \begin{lstlisting}
wget https://raw.githubusercontent.com/abhinavydv/AI1110-Assignments/master/RandomVariablesManual/codes/pdf_plot_tri.py
\end{lstlisting}
          and run it as follows to get \autoref{fig:tri_pdf}
          \begin{lstlisting}
$ python cdf_plot_tri.py
\end{lstlisting}
          \begin{figure}
              \centering
              \includegraphics[width=\columnwidth]{./figs/tri_pdf}
              \caption{The PDF of $T$}
              \label{fig:tri_pdf}
          \end{figure}

    \item Find the theoretical expressions for the PDF and CDF of $T$.\\
          \solution We have,
          \begin{align}
              T = U_1 + U_2
              \intertext{PDF can be found as follows}
              f_T(t) & = f_{U_1+U_2}(t)                                            \\
                     & = f_{U_1}(t)*f_{U_2}(t)                                     \\
                     & = \int_{-\infty}^{\infty} f_{U_1}(\tau)f_{U_2}(t-\tau)d\tau \\
                     & = \int_0^1f_{U_2}(t-\tau)d\tau                              \\
                     & = \begin{cases}
                  \int_0^1 0d\tau,   & t<0           \\
                  \int_0^t 1 d\tau,  & 0 \le t \le 1 \\
                  \int_{t-1}^1d\tau, & 1<t \le 2     \\
                  \int_0^1 0d\tau,   & t>2
              \end{cases}                                \\
                     & = \begin{cases}
                  t,   & 0 \le t \le 1    \\
                  2-t, & 1 \le t \le 2    \\
                  0,   & \text{otherwise}
              \end{cases}
              \intertext{Now, The CDF can be determined as follows,}
              F_T(x) & = \int_{-\infty}^x f_T(t)dt                                 \\
                     & = \begin{cases}
                  0,                               & x<0          \\
                  \int_0^x tdt,                    & 0\le x \le 1 \\
                  \int_0^1 tdt + \int_1^x (2-t)dt, & 1 < x \le 2  \\
                  1,                               & x>2          \\
              \end{cases}                                \\
                     & = \begin{cases}
                  0,                  & x < 0         \\
                  \frac{x^2}{2},      & 0 \le x \le 1 \\
                  2x-\frac{x^2}{2}-1, & 1 < x \le 2   \\
                  1,                  & x > 2
              \end{cases}
          \end{align}


    \item Verify your results through a plot.\\
          \solution
          Already plotted in \autoref{fig:tri_cdf} and \autoref{fig:tri_pdf}

\end{enumerate}
\section{Maximul Likelihood}
\begin{enumerate}[label=\thesection.\arabic*
        ,ref=\thesection.\theenumi]
    \item Generate
          \begin{equation}
              Y = AX+N,
          \end{equation}
          where $A = 5 \text{ dB}, X \in \cbrak{1,-1}$,  is Bernoulli and $N \sim \gauss{0}{1}$.
    \item Plot $Y$.
    \item Guess how to estimate $X$ from $Y$.
    \item
          \label{ml-ch4_sim}
          Find
          \begin{equation}
              P_{e|0} = \pr{\hat{X} = -1|X=1}
          \end{equation}
          and
          \begin{equation}
              P_{e|1} = \pr{\hat{X} = 1|X=-1}
          \end{equation}
          %
    \item Find $P_e$.
          %
    \item
          Verify by plotting  the theoretical $P_e$.
\end{enumerate}
\section{Gaussian to Other}
\begin{enumerate}[label=\thesection.\arabic*
        ,ref=\thesection.\theenumi]
    \item
          Let $X_1 \sim  \gauss{0}{1}$ and $X_2 \sim  \gauss{0}{1}$. Plot the CDF and PDF of
          %
          \begin{equation}
              V = X_1^2 + X_2^2
          \end{equation}
          %
          %
          %
    \item
          If
          %
          \begin{equation}
              F_{V}(x) =
              \begin{cases}
                  1 - e^{-\alpha x} & x \geq 0 \\
                  0                 & x < 0,
              \end{cases}
          \end{equation}
          %
          find $\alpha$.
          %
    \item
          \label{ch3_raleigh_sim}
          Plot the CDF and PDf of
          %
          \begin{equation}
              A = \sqrt{V}
          \end{equation}
          %
\end{enumerate}
\section{Conditional Probability}
\begin{enumerate}[label=\thesection.\arabic*
        ,ref=\thesection.\theenumi]
    \item
    \item
          \label{ch4_sim}
          Plot
          \begin{equation}
              P_e = \pr{\hat{X} = -1|X=1}
          \end{equation}
          %
          for
          \begin{equation}
              Y = AX+N,
          \end{equation}
          where $A$ is Raleigh with $E\sbrak{A^2} = \gamma, N \sim \gauss{0}{1}, X \in \brak{-1,1}$ for $0 \le \gamma \le 10$ dB.
          %
    \item
          Assuming that $N$ is a constant, find an expression for $P_e$.  Call this $P_e(N)$
          %
    \item
          %
          \label{ch4_anal}
          For a function $g$,
          \begin{equation}
              E\sbrak{g(X)} = \int_{-\infty}^{\infty}g(x)p_{X}(x)\, dx
          \end{equation}
          %
          Find $P_e = E\sbrak{P_e(N)}$.
          %
    \item
          Plot $P_e$ in problems \ref{ch4_sim} and \ref{ch4_anal} on the same graph w.r.t $\gamma$.  Comment.
\end{enumerate}
\section{Two Dimensions}
Let
\begin{equation}
    \mbf{y} = A\mbf{x} + \mbf{n},
\end{equation}
where
\begin{align}
    x       & \in \brak{\mbf{s}_0,\mbf{s}_1},
    \mbf{s}_0 =
    \begin{pmatrix}
        1
        \\
        0
    \end{pmatrix},
    \mbf{s}_1 =
    \begin{pmatrix}
        0
        \\
        1
    \end{pmatrix}
    \\
    \mbf{n} & =
    \begin{pmatrix}
        n_1
        \\
        n_2
    \end{pmatrix},
    n_1,n_2 \sim \gauss{0}{1}.
\end{align}
%
\begin{enumerate}[label=\thesection.\arabic*
        ,ref=\thesection.\theenumi]
    %%
    \item
          \label{ch5_fsk}
          Plot
          %
          \begin{equation}
              \mbf{y}|\mbf{s}_0 \text{ and } \mbf{y}|\mbf{s}_1
          \end{equation}
          %
          on the same graph using a scatter plot.
          %
    \item
          For the above problem, find a decision rule for detecting the symbols $\mbf{s}_0 $ and $\mbf{s}_1$.
          %
    \item
          Plot
          \begin{equation}
              P_e = \pr{\hat{\mbf{x}} = \mbf{s}_1|\mbf{x} = \mbf{s}_0}
          \end{equation}
          with respect to the SNR from 0 to 10 dB.
          %
    \item
          Obtain an expression for $P_e$. Verify this by comparing the theory and simulation plots on the same graph.
          %
\end{enumerate}

\end{document}

