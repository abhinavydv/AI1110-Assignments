\let\negmedspace\undefined
\let\negthickspace\undefined
\documentclass[journal,11pt,twocolumn]{IEEEtran}
\usepackage{gensymb}
\usepackage{amssymb}
\usepackage[cmex10]{amsmath}
\usepackage{amsthm}
\usepackage[export]{adjustbox}
\usepackage{bm}
\usepackage{longtable}
\usepackage{enumitem}
\usepackage{mathtools}
\usepackage{tikz}
\usepackage[breaklinks=true]{hyperref}
\usepackage{listings}
\usepackage{color}                                            %%
\usepackage{array}                                            %%
\usepackage{longtable}                                        %%
\usepackage{calc}                                             %%
\usepackage{multirow}                                         %%
\usepackage{hhline}                                           %%
\usepackage{ifthen}                                           %%
\usepackage{lscape}    
\usepackage{multicol}
% \usepackage{biblatex}
% \usepackage{enumerate}
\DeclareMathOperator*{\Res}{Res}
\renewcommand\thesection{\arabic{section}}
\renewcommand\thesubsection{\thesection.\arabic{subsection}}
\renewcommand\thesubsubsection{\thesubsection.\arabic{subsubsection}}
\renewcommand\thesectiondis{\arabic{section}}
\renewcommand\thesubsectiondis{\thesectiondis.\arabic{subsection}}
\renewcommand\thesubsubsectiondis{\thesubsectiondis.\arabic{subsubsection}}
\hyphenation{op-tical net-works semi-conduc-tor}
\def\inputGnumericTable{}                                 %%
\lstset{
frame=single, 
breaklines=true,
columns=fullflexible
}
\begin{document}
    \newtheorem{theorem}{Theorem}[section]
    \newtheorem{problem}{Problem}
    \newtheorem{proposition}{Proposition}[section]
    \newtheorem{lemma}{Lemma}[section]
    \newtheorem{corollary}[theorem]{Corollary}
    \newtheorem{example}{Example}[section]
    \newtheorem{definition}[problem]{Definition}
    \newcommand{\BEQA}{\begin{eqnarray}}
    \newcommand{\EEQA}{\end{eqnarray}}
    \newcommand{\define}{\stackrel{\triangle}{=}}
    \newcommand*\circled[1]{\tikz[baseline=(char.base)]{
        \node[shape=circle,draw,inner sep=2pt] (char) {#1};}}
    \bibliographystyle{IEEEtran}
    \providecommand{\mbf}{\mathbf}
    \providecommand{\pr}[1]{\ensuremath{\Pr\left(#1\right)}}
    \providecommand{\qfunc}[1]{\ensuremath{Q\left(#1\right)}}
    \providecommand{\sbrak}[1]{\ensuremath{{}\left[#1\right]}}
    \providecommand{\lsbrak}[1]{\ensuremath{{}\left[#1\right.}}
    \providecommand{\rsbrak}[1]{\ensuremath{{}\left.#1\right]}}
    \providecommand{\brak}[1]{\ensuremath{\left(#1\right)}}
    \providecommand{\lbrak}[1]{\ensuremath{\left(#1\right.}}
    \providecommand{\rbrak}[1]{\ensuremath{\left.#1\right)}}
    \providecommand{\cbrak}[1]{\ensuremath{\left\{#1\right\}}}
    \providecommand{\lcbrak}[1]{\ensuremath{\left\{#1\right.}}
    \providecommand{\rcbrak}[1]{\ensuremath{\left.#1\right\}}}
    \theoremstyle{remark}
    \newtheorem{rem}{Remark}
    \newcommand{\sgn}{\mathop{\mathrm{sgn}}}
    \providecommand{\abs}[1]{\left\vert#1\right\vert}
    \providecommand{\res}[1]{\Res\displaylimits_{#1}} 
    \providecommand{\norm}[1]{\left\lVert#1\right\rVert}
    %\providecommand{\norm}[1]{\lVert#1\rVert}
    \providecommand{\mtx}[1]{\mathbf{#1}}
    \providecommand{\mean}[1]{E\left[ #1 \right]}
    \providecommand{\fourier}{\overset{\mathcal{F}}{ \rightleftharpoons}}
    %\providecommand{\hilbert}{\overset{\mathcal{H}}{ \rightleftharpoons}}
    \providecommand{\system}{\overset{\mathcal{H}}{ \longleftrightarrow}}
        %\newcommand{\solution}[2]{\textbf{Solution:}{#1}}
    \newcommand{\solution}{\noindent \textbf{Solution: }}
    \newcommand{\question}[1]{\noindent \textbf{#1}}
    \newcommand{\cosec}{\,\text{cosec}\,}
    \providecommand{\dec}[2]{\ensuremath{\overset{#1}{\underset{#2}{\gtrless}}}}
    \newcommand{\myvec}[1]{\ensuremath{\begin{pmatrix}#1\end{pmatrix}}}
    \newcommand{\mydet}[1]{\ensuremath{\begin{vmatrix}#1\end{vmatrix}}}
    \newcommand*{\permcomb}[4][0mu]{{{}^{#3}\mkern#1#2_{#4}}}
    \newcommand*{\perm}[1][-3mu]{\permcomb[#1]{P}}
    \newcommand*{\comb}[1][-1mu]{\permcomb[#1]{C}}
    \makeatletter
    \@addtoreset{figure}{problem}
    \makeatother
    \let\StandardTheFigure\thefigure
    \let\vec\mathbf
    \def\putbox#1#2#3{\makebox[0in][l]{\makebox[#1][l]{}\raisebox{\baselineskip}[0in][0in]{\raisebox{#2}[0in][0in]{#3}}}}
        \def\rightbox#1{\makebox[0in][r]{#1}}
        \def\centbox#1{\makebox[0in]{#1}}
        \def\topbox#1{\raisebox{-\baselineskip}[0in][0in]{#1}}
        \def\midbox#1{\raisebox{-0.5\baselineskip}[0in][0in]{#1}}
    \vspace{3cm}
    \title{AI1110 Assignment 6}
    \author{Abhinav Yadav (cs21btech11002)}
    % make the title area
    \maketitle
    \newpage

    \question{[CBSE class 12 Chapter 13 example 4]}\\
    In a school, there are 1000 students, out of which 430 are girls. It is known
that out of 430, 10\% of the girls study in class XII. What is the probability that a student
chosen randomly studies in Class XII given that the chosen student is a girl?

    \solution\\
    % Let $X\in \cbrak{0, 1}$ denote the gender of the student where $X=0$ represents a girl and
    % $X=1$ represents a boy. And let $Y\in\cbrak{0, 1, 2 \dots 12}$ denote the class of student
    % where $Y=i$ represents that student is in class i.\\
    % Then,
    % \begin{align}
    %     \pr{X=0} &= \frac{430}{1000} = \frac{43}{100}\\
    %     \pr{X=0, Y=12} &= \frac{43}{1000}
    % \end{align}
    See \autoref{tb:desc} and \autoref{tb:prob} for input probabilities

    \begin{table}[h]
        \begin{tabular}{|c|c|}
            \hline
            Events & Description\\
            \hline
            X=0 & Student is girl\\
            \hline
            X=1 & Student is boy\\
            \hline
            Y=12 & Student is in class 12\\
            \hline
        \end{tabular}\\
        \caption{}
        \label{tb:desc}
    \end{table}
    \begin{table}[h]
        \begin{tabular}{|c|c|}
            \hline
            Probability & Value\\
            \hline
            \pr{X=0} & $\frac{430}{1000} = \frac{43}{100}$\\
            \hline
            \pr{Y=12, X=0} & $\frac{43}{1000}$\\
            \hline
            \pr{Y=12 | X=0} & ?\\
            \hline
        \end{tabular}\\
        \caption{}
        \label{tb:prob}
    \end{table}
    The desired probability is obtained using \autoref{tb:prob} as follows
    \begin{align}
        \pr{Y=12 | X=0} &= \frac{\pr{Y=12, X=0}}{\pr{X=0}}\\
        \pr{Y=12 | X=0} &= \frac{\frac{43}{1000}}{\frac{43}{100}} = \frac{1}{10}\\
    \end{align}

    
\end{document}

