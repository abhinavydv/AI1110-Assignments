\documentclass{beamer}
\usetheme{Madrid}
% \usetheme{Frankfurt}
% \usetheme{Darmstadt}
% \usetheme{Berlin}
% \usetheme{Warsaw}
% \usetheme{Berkeley}
% \usetheme{Bergen}
% \usetheme{CambridgeUS}
% \usetheme{Copenhagen}

\providecommand{\pr}[1]{\ensuremath{\Pr\left(#1\right)}}
\providecommand{\cbrak}[1]{\ensuremath{\left\{#1\right\}}}
\newcommand{\myvec}[1]{\ensuremath{\begin{pmatrix}#1\end{pmatrix}}}
\newcommand{\convolve}{\oplus}

\title{AI1110 - Probability and Random Variables \\
        Assignment 8}
\author{Abhinav Yadav \\ 
        cs21btech11002}

\begin{document}
    \maketitle
    % \begin{frame}{Table of Contents}
    %     \tableofcontents
    % \end{frame}

    \begin{frame}{Papoulis Chapter 4}
        \section{Question}
        \begin{block}{Q 4-30}
            The probability that a driver will have an accident in 1 month equals 0.02. 
            Find the probability that in 100 months he will have three accidents.
        \end{block}
    \end{frame}

    \begin{frame}[allowframebreaks]{Solution}
        \section{Solution}
        \begin{block}{}
            Let the event of accident be represented by a random variable X,
            where X=0 represents that the driver will not have an accident and 
            X=1 represents that the driver will have an accident.\\
            Given,
            \begin{align}
                \Pr(X=0) = 0.98\\
                \Pr(X=1) = 0.02
            \end{align}
            This distribution can be represented using a matrix as follows
            \begin{align}
                H = \myvec{0.98 \\ 0.02}
            \end{align}
            Let the number of accidents in 100 months be represented by $Y\in\cbrak{0, 1, 2, 3, ... 100}$.\\
        \end{block}
        
        \begin{block}{}
            
            The matrix
            \begin{align}
                P = \myvec{\Pr(Y=0) \\ \Pr(Y=1) \\ \Pr(Y=2) \\ ... \\ \Pr(Y=100)}
                \intertext{can be obtained by the following convolution}
                P = Y \convolve Y \convolve Y \convolve .. Y (100 times)
                \intertext{Using python for the above calculation, we get}
                \Pr(Y=3) = 0.18227
            \end{align}
        \end{block}

    \end{frame}
\end{document}
