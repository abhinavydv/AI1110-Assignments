\documentclass{beamer}
\usetheme{Madrid}
% \usetheme{Frankfurt}
% \usetheme{Darmstadt}
% \usetheme{Berlin}
% \usetheme{Warsaw}
% \usetheme{Berkeley}
% \usetheme{Bergen}
% \usetheme{CambridgeUS}
% \usetheme{Copenhagen}

\providecommand{\pr}[1]{\ensuremath{\Pr\left(#1\right)}}
\providecommand{\cbrak}[1]{\ensuremath{\left\{#1\right\}}}
\newcommand{\myvec}[1]{\ensuremath{\begin{pmatrix}#1\end{pmatrix}}}
\newcommand{\convolve}{\oplus}

\title{AI1110 - Probability and Random Variables \\
        Assignment 9}
\author{Abhinav Yadav \\ 
        cs21btech11002}

\begin{document}
\maketitle
% \begin{frame}{Table of Contents}
%     \tableofcontents
% \end{frame}

\begin{frame}{Papoulis Chapter 8}
    \section{Question}
    \begin{block}{Q 8-12}
        In a market survey, it was reported that 29\% of respondents favor product A. The poll was
        conducted with confidence coefficient 0.95, and the margin of error was $\pm$4\%. Find the
        number of respondents.
    \end{block}
\end{frame}

\begin{frame}[allowframebreaks]{Solution}
    \section{Solution}
    \begin{block}{}
        The formula for interval estimate of probability is
        \begin{align}
            p        & \approx \bar x \pm z_u \sqrt{\frac{\bar x(1-\bar x)}{n}} \\
            \label{eq:err}
            \Delta p & = z_u \sqrt{\frac{\bar x(1-\bar x)}{n}}                  \\
            \intertext{Given,}
            \bar x   & = 0.29                                                   \\
            \Delta p & = 0.04                                                   \\
            \gamma   & = 0.95                                                   \\
            n        & = ?
        \end{align}
    \end{block}
    \pagebreak
    \begin{block}
        \pagebreak
        \begin{align}
            \intertext{Therefore,}
            u             & = \frac{\gamma+1}{2} = 0.975     \\
            z_u           & = z_{0.975} = 2                  \\
            \intertext{Putting the values in equation \autoref{eq:err}, we get}
            0.04          & = 2\sqrt{\frac{0.29(1-0.29)}{n}} \\
            \Rightarrow n & \ge 514.75                       \\
            \Rightarrow n & \ge 515
        \end{align}
    \end{block}

\end{frame}
\end{document}
